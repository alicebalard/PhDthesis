Genetic diversity in animal hybrids can affect each physiological system differently. If reproduction usually suffers from breakdown of coadapted complexes, resistance to parasite could benefit from the novelty brought by recombination. The question of hybrid relative resistance or susceptibility to parasites in the European house mouse hybrid zone has been discussed for the past thirty years, leading to contradictory conclusions on relative hybrid fitness. But drawing conclusions on hybrid host fitness in relation to parasites requires first to investigate the link between resistance and host health. Resistance (the host’s capacity to reduce parasite burden) and tolerance (the host’s capacity to reduce impact on host health of a given parasite burden) manifest two different lines of immune defences. Trade-offs arise, as resistance limits infection load and thereby the scope of possible tolerance, and both resistance and tolerance can be costly in terms of resource allocation. \\
During this PhD project, we assessed infections by intracellular protozoans, \textit{Eimeria} spp., using field sampling and laboratory infection of wild and wild-derived mice from a hybrid zone between Mus musculus domesticus and Mus musculus musculus. We asked (1) whether hybrid mice are more or less resistant than their parents and (2) how resistance and tolerance are correlated, this correlation potentially differing between \textit{Eimeria} species. We found lower intensities in hybrid hosts than in parental mice and no evidence of lowered probability of infection or increased mortality in the centre of the hybrid zone. This challenges the longstanding impression that hybrid mice are more highly parasitised than parentals. Upon experimental infection, we found a trade-off between resistance and tolerance in \textit{E. falciformis}, but not in \textit{E. ferrisi}. Building on previous research showing that resistance and tolerance should be studied jointly, our results show that assumptions on coupling of the two can not be transferred across even closely related parasite taxa. We showed that the impact of parasitism on hybrid fitness is a complex matter that needs to be investigated for each parasite beyond the measurement of hybrid vigour on resistance, taking into account possible trade-offs between resistance and tolerance.
