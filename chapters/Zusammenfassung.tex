\textbf{Resistenz und Toleranz gegenüber \textit{Eimeria} in der europäischen Hausmaus-Hybridzone} \\

Die genetische Vielfalt von Tierhybriden kann jedes physiologische System unterschiedlich beeinflussen. Auch wenn die Fortpflanzung in der Regel unter dem Abbau koadaptierter Komplexe leidet, könnte die Resistenz gegen Parasiten gleichzeitig von der Neuheit profitieren, die die Rekombination mit sich bringt. Die Frage der relativen Hybridresistenz oder der Anfälligkeit für Parasiten in der europäischen Hausmaus-Hybridzone wird seit dreißig Jahren diskutiert, was zu widersprüchlichen Schlussfolgerungen über die relative Hybridfitness geführt hat. Um jedoch Schlussfolgerungen über die Fitness von Hybriden in Bezug auf Parasiten ziehen zu können, muss zunächst der Zusammenhang zwischen Resistenz und Gesundheit des Wirts untersucht werden. Resistenz (die Fähigkeit des Wirtes, die Parasitenlast zu reduzieren) und Toleranz (die Fähigkeit des Wirtes, die Auswirkungen einer gegebenen Parasitenlast auf die Gesundheit des Wirtes zu reduzieren) manifestieren zwei verschiedene Linien der Immunabwehr. Es kommt zu Kompromissen, da die Resistenz die Infektionslast und damit den Umfang der möglichen Toleranz begrenzt und sowohl Resistenz als auch Toleranz im Hinblick auf die Ressourcenallokation kostspielig sein können. \\
Während dieses Dissertationsprojekts untersuchten wir Infektionen durch intrazelluläre Protozoen, \textit{Eimeria} spp., anhand von Feldproben und Laborinfektionen von wilden und ursprünglich aus der Wildnis stammenden Mäusen aus einer Hybridzone zwischen \textit{Mus musculus domesticus} und \textit{Mus musculus musculus}. Wir fragten: (1) ob Hybridmäuse mehr oder weniger resistent als ihre Elterntiere sind und (2) in welcher Form Resistenz und Toleranz korrelieren, und ob diese Korrelation sich bei unterschiedlichen \textit{Eimeria}-Arten verändert. Wir fanden niedrigere Intensitäten in hybriden Wirten als in elterlichen Mäusen und keinen Hinweis auf eine verminderte Infektionswahrscheinlichkeit oder erhöhte Mortalität im Zentrum der Hybridzone. Dies stellt den seit langem bestehenden Eindruck in Frage, dass Hybridmäuse stärker parasitiert werden als Elterntiere. Bei der experimentellen Infektion fanden wir einen Kompromiss zwischen Resistenz und Toleranz bei \textit{E. falciformis}, aber nicht bei \textit{E. ferrisi}. Aufbauend auf früheren Forschungsarbeiten, die gezeigt haben, dass Resistenz und Toleranz gemeinsam untersucht werden sollten, zeigen unsere Ergebnisse, dass die Annahmen zur Kopplung der beiden nicht einmal auf eng verwandte Parasitentaxa übertragen werden können. Wir zeigten, dass der Einfluss des Parasitismus auf die Fitness von Hybriden eine komplexe Angelegenheit ist, die für jeden Parasiten über die Messung der Hybridkraft auf die Resistenz hinaus untersucht werden muss, wobei mögliche Kompromisse zwischen Resistenz und Toleranz berücksichtigt werden müssen.
